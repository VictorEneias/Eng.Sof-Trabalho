\documentclass[12pt,a4paper]{article}
\usepackage[utf8]{inputenc}
\usepackage[portuguese]{babel}
\usepackage{amsmath}
\usepackage{amsfonts}
\usepackage{amssymb}
\usepackage{graphicx}
\usepackage{geometry}
\usepackage{hyperref}
\usepackage{titlesec}
\usepackage{enumitem}
\usepackage{fancyhdr}

% Configurações de página
\geometry{margin=2.5cm}
\pagestyle{fancy}
\fancyhf{}
\rhead{Sistema de Gestão de Feiras}
\lhead{Documento de Visão e Escopo}
\cfoot{\thepage}

% Configurações de títulos
\titleformat{\section}{\Large\bfseries}{\thesection}{1em}{}
\titleformat{\subsection}{\large\bfseries}{\thesubsection}{1em}{}
\titleformat{\subsubsection}{\normalsize\bfseries}{\thesubsubsection}{1em}{}

\begin{document}

% Página de título
\begin{titlepage}
\centering
\vspace*{2cm}

{\Huge\bfseries Documento de Visão e Escopo}

\vspace{0.5cm}

{\LARGE Sistema de Gestão de Feiras}

\vspace{2cm}

{\large Higor Roger de Freitas Santos - 221006440\\
Victor Eneias Oliveira - 221038364}

\vspace{1cm}

{\large Engenharia de Software CIC0105 Turma 01 2025.1}

\vfill

{\large Versão: 1.0\\
Data: Junho 2025\\
Documento: Visão e Escopo - Sistema de Gestão de Feiras}

\end{titlepage}

\newpage
\tableofcontents
\newpage

\section{Objetivo do Sistema}

O Sistema de Gestão de Feiras é uma plataforma web desenvolvida para facilitar a organização, administração e participação em eventos comerciais do tipo feira. O sistema permite que organizadores criem e gerenciem eventos, expositores se cadastrem para participar apresentando seus produtos, e visitantes adquiram ingressos de acesso aos eventos.

A plataforma centraliza todas as operações relacionadas a feiras comerciais em uma única interface digital, proporcionando controle total do ciclo de vida do evento desde sua criação até a emissão de ingressos para visitantes.

\section{Problema que Resolve}

O sistema resolve a complexidade e fragmentação na gestão de eventos do tipo feira, onde tradicionalmente:

\begin{itemize}
    \item \textbf{Organizadores} enfrentam dificuldades para centralizar informações sobre o evento, controlar participantes e gerenciar ingressos
    \item \textbf{Expositores} precisam lidar com processos manuais ou sistemas isolados para se cadastrar em eventos e apresentar seus produtos
    \item \textbf{Visitantes} não possuem uma forma padronizada e digital de adquirir ingressos e conhecer previamente os expositores e produtos disponíveis
    \item \textbf{Administração geral} carece de visibilidade integrada sobre todos os aspectos do evento
\end{itemize}

A ausência de uma plataforma unificada resulta em processos ineficientes, informações dispersas e experiência fragmentada para todos os envolvidos no ecossistema de feiras comerciais.

\section{Justificativa do Projeto}

O desenvolvimento deste sistema é justificado pelos seguintes fatores:

\subsection{Digitalização de Processos}
A crescente necessidade de digitalização de eventos presenciais, acelerada por mudanças no comportamento do consumidor e demandas por eficiência operacional.

\subsection{Centralização de Informações}
Eliminação da dispersão de dados entre planilhas, sistemas isolados e processos manuais, proporcionando uma fonte única de verdade para todos os stakeholders.

\subsection{Melhoria da Experiência do Usuário}
Simplificação dos processos para organizadores, expositores e visitantes através de interfaces intuitivas e fluxos otimizados.

\subsection{Controle e Segurança}
Implementação de autenticação robusta e autorização baseada em propriedade, garantindo que apenas usuários autorizados possam modificar informações específicas.

\subsection{Escalabilidade e Padronização}
Criação de uma base tecnológica moderna que pode ser replicada para diferentes tipos de feira e expandida conforme necessidades futuras.

\section{Stakeholders}

\subsection{Stakeholders Primários}

\subsubsection{Organizadores de Feira}
\begin{itemize}
    \item \textbf{Perfil}: Empresas, entidades ou pessoas responsáveis pela criação e gestão de eventos
    \item \textbf{Necessidades}: Criar feiras, definir informações do evento (datas, local, descrição), controlar expositores participantes e monitorar ingressos emitidos
    \item \textbf{Benefícios}: Visão centralizada do evento, controle total sobre participantes e automação do processo de gestão
\end{itemize}

\subsubsection{Expositores}
\begin{itemize}
    \item \textbf{Perfil}: Empresas, empreendedores ou artesãos que desejam participar de feiras para apresentar produtos ou serviços
    \item \textbf{Necessidades}: Cadastrar-se em feiras de interesse, apresentar informações da empresa, exibir catálogo de produtos com preços
    \item \textbf{Benefícios}: Processo simplificado de participação, vitrine digital para produtos, maior visibilidade
\end{itemize}

\subsubsection{Visitantes}
\begin{itemize}
    \item \textbf{Perfil}: Público geral interessado em visitar feiras, conhecer produtos e adquirir ingressos
    \item \textbf{Necessidades}: Visualizar feiras disponíveis, conhecer expositores e produtos, adquirir ingressos de acesso
    \item \textbf{Benefícios}: Acesso digital a informações do evento, processo simplificado de aquisição de ingressos, conhecimento prévio do que encontrará no evento
\end{itemize}

\subsection{Stakeholders Secundários}

\subsubsection{Desenvolvedores}
\begin{itemize}
    \item \textbf{Perfil}: Equipe técnica responsável pela manutenção e evolução do sistema
    \item \textbf{Necessidades}: Código bem estruturado, documentação técnica, arquitetura escalável
    \item \textbf{Benefícios}: Base de código organizada com separação clara de responsabilidades
\end{itemize}

\subsubsection{Equipe de Infraestrutura}
\begin{itemize}
    \item \textbf{Perfil}: Profissionais responsáveis pela operação e monitoramento do sistema em produção
    \item \textbf{Necessidades}: Sistema estável, logs adequados, facilidade de deployment
    \item \textbf{Benefícios}: Aplicação com dependências bem definidas e configuração simplificada
\end{itemize}

\subsubsection{Auditores e Compliance}
\begin{itemize}
    \item \textbf{Perfil}: Responsáveis por verificar conformidade com regulamentações e segurança
    \item \textbf{Necessidades}: Logs de auditoria, controles de acesso, proteção de dados pessoais
    \item \textbf{Benefícios}: Sistema com autenticação robusta e autorização baseada em propriedade
\end{itemize}

\section{Funcionalidades de Alto Nível}

\subsection{Gestão de Usuários}
\begin{itemize}
    \item Registro de novos usuários com validação de email único
    \item Autenticação segura via credenciais email/senha
    \item Geração e validação de tokens JWT para sessões
\end{itemize}

\subsection{Gestão de Feiras}
\begin{itemize}
    \item Criação de eventos com informações completas (nome, descrição, datas, localização)
    \item Listagem pública de feiras disponíveis
    \item Edição e exclusão de feiras pelos organizadores criadores
\end{itemize}

\subsection{Gestão de Expositores}
\begin{itemize}
    \item Cadastro de expositores em feiras específicas
    \item Armazenamento de informações empresariais e contato
    \item Controle de participação por feira
\end{itemize}

\subsection{Gestão de Produtos}
\begin{itemize}
    \item Cadastro de produtos pelos expositores
    \item Informações detalhadas incluindo nome, descrição e preço
    \item Catálogo digital associado a cada expositor
\end{itemize}

\subsection{Gestão de Ingressos}
\begin{itemize}
    \item Emissão de ingressos únicos para feiras específicas
    \item Controle de data de emissão
    \item Associação de ingressos a usuários compradores
\end{itemize}

\subsection{Controle de Autorização}
\begin{itemize}
    \item Restrição de edição/exclusão apenas para criadores de conteúdo
    \item Validação de propriedade em todas as operações sensíveis
    \item Proteção contra modificações não autorizadas
\end{itemize}

\section{Restrições e Dependências}

\subsection{Tecnológicas}

\subsubsection{Backend}
\begin{itemize}
    \item \textbf{FastAPI}: Framework web Python para desenvolvimento da API REST
    \item \textbf{SQLAlchemy}: ORM para mapeamento objeto-relacional e abstração do banco de dados
    \item \textbf{Pydantic}: Biblioteca para validação de dados e serialização
    \item \textbf{PassLib[bcrypt]}: Criptografia de senhas com algoritmo BCrypt
    \item \textbf{Python-JOSE}: Implementação de tokens JWT para autenticação stateless
\end{itemize}

\subsubsection{Frontend}
\begin{itemize}
    \item \textbf{React.js}: Biblioteca JavaScript para construção da interface de usuário
    \item \textbf{JavaScript ES6+}: Linguagem de programação para lógica do frontend
    \item \textbf{Fetch API}: Comunicação HTTP nativa do navegador para integração com backend
\end{itemize}

\subsubsection{Banco de Dados}
\begin{itemize}
    \item \textbf{SQLite}: Sistema de banco de dados relacional embarcado para desenvolvimento e testes
    \item \textbf{Integridade Referencial}: Chaves estrangeiras garantindo consistência entre entidades
\end{itemize}

\subsection{Segurança e Autenticação}
\begin{itemize}
    \item \textbf{JWT (JSON Web Tokens)}: Tokens com expiração de 60 minutos
    \item \textbf{CORS (Cross-Origin Resource Sharing)}: Configurado para permitir acesso do frontend
    \item \textbf{Autorização por Propriedade}: Apenas criadores podem modificar seus próprios registros
\end{itemize}

\subsection{Infraestrutura Mínima}
\begin{itemize}
    \item \textbf{Python 3.8+}: Runtime mínimo para execução do backend
    \item \textbf{Navegador Web Moderno}: Suporte a ES6+ e Fetch API para frontend
    \item \textbf{Servidor Web}: Uvicorn ou similar para servir aplicação FastAPI
    \item \textbf{Armazenamento}: Espaço para banco SQLite e arquivos da aplicação
\end{itemize}

\subsection{Dependências Operacionais}
\begin{itemize}
    \item \textbf{Conectividade de Rede}: Para comunicação entre frontend e backend
    \item \textbf{Persistência de Dados}: Banco de dados para armazenamento permanente
    \item \textbf{Gerenciamento de Sessão}: LocalStorage do navegador para tokens JWT
\end{itemize}

\section{Contexto do Domínio}

\subsection{Conceitos Fundamentais}

\subsubsection{Feira}
Uma \textbf{feira} é um evento comercial temporário caracterizado por:
\begin{itemize}
    \item \textbf{Duração limitada}: Possui data de início e fim bem definidas
    \item \textbf{Local físico}: Acontece em espaço específico (centro de convenções, praças, pavilhões)
    \item \textbf{Finalidade comercial}: Visa promover produtos, serviços ou conhecimento
    \item \textbf{Organização centralizada}: Possui um responsável que coordena o evento
\end{itemize}

\textbf{Exemplo prático}: ``Feira de Tecnologia São Paulo 2025'' acontecendo de 15 a 17 de junho no Centro de Convenções Rebouças, organizada pela empresa TechEvents.

\subsubsection{Expositor}
Um \textbf{expositor} é uma entidade (empresa, empreendedor, artesão) que participa da feira para:
\begin{itemize}
    \item \textbf{Apresentar produtos ou serviços}: Demonstrar seu portfólio ao público
    \item \textbf{Estabelecer contatos comerciais}: Networking com visitantes e outros expositores
    \item \textbf{Gerar vendas}: Comercializar diretamente ou captar leads
    \item \textbf{Divulgar marca}: Aumentar visibilidade e reconhecimento
\end{itemize}

\textbf{Exemplo prático}: A empresa ``TechSolutions'' se cadastra como expositora na Feira de Tecnologia para apresentar seus softwares de gestão empresarial.

\subsubsection{Produto}
Um \textbf{produto} representa itens oferecidos pelos expositores:
\begin{itemize}
    \item \textbf{Tangíveis}: Equipamentos, artesanatos, alimentos, vestuário
    \item \textbf{Intangíveis}: Softwares, serviços, consultorias
    \item \textbf{Informações essenciais}: Nome, descrição detalhada e preço
    \item \textbf{Catálogo digital}: Permite visitantes conhecerem a oferta antes da visita
\end{itemize}

\textbf{Exemplo prático}: A TechSolutions cadastra o produto ``ERP Empresarial v2.0'' com descrição ``Sistema integrado de gestão'' e preço ``R\$ 5.000,00''.

\subsubsection{Visitante e Ingresso}
\textbf{Visitantes} são o público-alvo das feiras, e \textbf{ingressos} controlam seu acesso:
\begin{itemize}
    \item \textbf{Identificação única}: Cada ingresso possui número exclusivo para controle
    \item \textbf{Associação ao evento}: Ingressos são específicos para uma feira
    \item \textbf{Controle de acesso}: Permite entrada apenas de pessoas autorizadas
    \item \textbf{Rastreabilidade}: Data de emissão e comprador são registrados
\end{itemize}

\textbf{Exemplo prático}: João Silva adquire o ingresso \#ABC123 para a Feira de Tecnologia, emitido em 10/06/2025, permitindo sua entrada no evento.

\subsection{Fluxo Operacional Típico}

\begin{enumerate}
    \item \textbf{Criação do Evento}: Organizador cadastra nova feira definindo todas as informações básicas
    \item \textbf{Divulgação}: Feira fica disponível publicamente para consulta
    \item \textbf{Adesão de Expositores}: Empresas interessadas se cadastram para participar
    \item \textbf{Cadastro de Produtos}: Expositores montam seus catálogos digitais
    \item \textbf{Venda de Ingressos}: Visitantes adquirem acesso ao evento
    \item \textbf{Realização da Feira}: Evento acontece conforme planejado
    \item \textbf{Controle de Acesso}: Ingressos são validados na entrada
\end{enumerate}

\subsection{Relacionamentos entre Entidades}

\begin{itemize}
    \item \textbf{Organizador → Feira}: Um organizador pode criar múltiplas feiras (1:N)
    \item \textbf{Feira → Expositor}: Uma feira pode ter vários expositores (1:N)
    \item \textbf{Expositor → Produto}: Um expositor pode oferecer múltiplos produtos (1:N)
    \item \textbf{Feira → Ingresso}: Uma feira pode ter vários ingressos emitidos (1:N)
    \item \textbf{Usuário → Múltiplos Papéis}: Um usuário pode ser organizador, expositor e visitante simultaneamente
\end{itemize}

Este contexto garante que o sistema atenda às necessidades reais do domínio de feiras comerciais, proporcionando uma solução completa e integrada para todos os stakeholders envolvidos.

\end{document}