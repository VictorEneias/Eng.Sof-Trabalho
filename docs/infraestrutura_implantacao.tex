\documentclass[12pt,a4paper]{article}
\usepackage[utf8]{inputenc}
\usepackage[portuguese]{babel}
\usepackage{geometry}
\usepackage{titlesec}
\usepackage{enumitem}

\geometry{margin=2.5cm}
\titleformat{\section}{\Large\bfseries}{\thesection}{1em}{}
\titleformat{\subsection}{\large\bfseries}{\thesubsection}{1em}{}

\title{\textbf{Descrição da Infraestrutura de Implantação}\\
\large Sistema de Gestão de Feiras (SGF)}
\author{Higor Roger de Freitas Santos \quad 221006440\\
Victor Eneias Oliveira \quad 221038364\\
\\
Engenharia de Software CIC0105 Turma 01 2025.1}
\date{\today}

\begin{document}

\maketitle
\tableofcontents
\newpage

\section{Hardware}

\subsection{Requisitos Mínimos}
\begin{itemize}
    \item \textbf{CPU:} 1 core, 1.0 GHz
    \item \textbf{RAM:} 2 GB
    \item \textbf{Armazenamento:} 5 GB
    \item \textbf{Rede:} Conexão com internet
\end{itemize}

\section{Software}

\subsection{Sistema Operacional}
\begin{itemize}
    \item Windows 10/11
    \item Linux (Ubuntu, CentOS)
    \item macOS
\end{itemize}

\subsection{Dependências Necessárias}
\begin{itemize}
    \item Python 3.8+
    \item Navegador web moderno (Chrome, Firefox, Edge)
\end{itemize}

\subsection{Bibliotecas Python}
Conforme arquivo \texttt{requirements.txt}:
\begin{itemize}
    \item FastAPI
    \item Uvicorn
    \item SQLAlchemy
    \item Pydantic
    \item python-jose
    \item passlib
    \item bcrypt
\end{itemize}

\section{Serviços}

\subsection{Banco de Dados}
\begin{itemize}
    \item SQLite (arquivo local)
    \item Criado automaticamente na primeira execução
    \item Localização: \texttt{./esw.db}
\end{itemize}

\subsection{Servidor Web}
\begin{itemize}
    \item Backend: Uvicorn (porta 8000)
    \item Frontend: Arquivos HTML/CSS/JS servidos diretamente
\end{itemize}

\section{Execução}

\subsection{Instalação}
\begin{enumerate}
    \item Instalar Python 3.8+
    \item Executar: \texttt{pip install -r requirements.txt}
    \item Executar: \texttt{uvicorn main:app --reload}
    \item Acessar: \texttt{http://localhost:8000}
\end{enumerate}

\subsection{Estrutura de Arquivos}
\begin{itemize}
    \item \texttt{main.py} - Aplicação principal
    \item \texttt{frontend/} - Arquivos da interface
    \item \texttt{routers/} - Endpoints da API
    \item \texttt{requirements.txt} - Dependências
    \item \texttt{esw.db} - Banco de dados (criado automaticamente)
\end{itemize}

\end{document} 