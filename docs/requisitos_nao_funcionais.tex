\documentclass[12pt,a4paper]{article}
\usepackage[utf8]{inputenc}
\usepackage[portuguese]{babel}
\usepackage{geometry}
\usepackage{titlesec}
\usepackage{enumitem}
\usepackage{xcolor}

\geometry{margin=2.5cm}
\titleformat{\section}{\Large\bfseries}{\thesection}{1em}{}
\titleformat{\subsection}{\large\bfseries}{\thesubsection}{1em}{}
\titleformat{\subsubsection}{\normalsize\bfseries}{\thesubsubsection}{1em}{}

\title{\textbf{Especificação de Requisitos Não Funcionais (RNFs)}\\
\large Sistema de Gestão de Feiras}
\author{Higor Roger de Freitas Santos - 221006440\\
Victor Eneias Oliveira - 221038364\\
\\
Engenharia de Software CIC0105 Turma 01 2025.1}
\date{\today}

\begin{document}

\maketitle

\tableofcontents
\newpage

\section{Requisitos de Segurança}

\subsection{RNF01 - Autenticação JWT}

\textbf{Descrição}: O sistema deve implementar autenticação usando tokens JWT para garantir a identidade dos usuários.

\textbf{Implementação}:
\begin{itemize}
    \item Tokens JWT gerados no login com expiração de 60 minutos
    \item Validação de token em endpoints protegidos
    \item Retorno de erro 401 para tokens inválidos ou expirados
\end{itemize}

\textbf{Critério de Aceitação}: Login bem-sucedido gera token válido que permite acesso a funcionalidades protegidas. Tokens expirados ou inválidos são rejeitados.

\subsection{RNF02 - Criptografia de Senhas}

\textbf{Descrição}: O sistema deve armazenar senhas de forma segura usando criptografia.

\textbf{Implementação}:
\begin{itemize}
    \item Senhas criptografadas com BCrypt antes do armazenamento
    \item Verificação segura durante o login
    \item Senhas nunca armazenadas em texto plano
\end{itemize}

\textbf{Critério de Aceitação}: Todas as senhas no banco de dados estão criptografadas. Login funciona corretamente com verificação de hash.

\subsection{RNF03 - Controle de Acesso}

\textbf{Descrição}: O sistema deve garantir que usuários só possam modificar recursos que criaram.

\textbf{Implementação}:
\begin{itemize}
    \item Verificação de propriedade antes de permitir edição ou exclusão
    \item Comparação entre usuário autenticado e criador do recurso
    \item Retorno de erro 403 para tentativas não autorizadas
\end{itemize}

\textbf{Critério de Aceitação}: Usuários conseguem modificar apenas seus próprios recursos. Tentativas de modificar recursos de outros usuários são bloqueadas.

\section{Requisitos de Desempenho}

\subsection{RNF04 - Tempo de Resposta}

\textbf{Descrição}: O sistema deve responder às requisições em tempo adequado para uso interativo.

\textbf{Implementação}:
\begin{itemize}
    \item Operações básicas (login, listagem) devem ser rápidas
    \item Interface responsiva durante carregamento
    \item Feedback visual para operações demoradas
\end{itemize}

\textbf{Critério de Aceitação}: Operações comuns completam em tempo razoável. Interface não trava durante uso normal.

\subsection{RNF05 - Usuários Simultâneos}

\textbf{Descrição}: O sistema deve suportar múltiplos usuários acessando simultaneamente.

\textbf{Implementação}:
\begin{itemize}
    \item Configuração de banco para acesso concorrente
    \item Autenticação independente por usuário
    \item Isolamento de dados entre usuários
\end{itemize}

\textbf{Critério de Aceitação}: Múltiplos usuários podem usar o sistema ao mesmo tempo sem conflitos ou erros.

\section{Requisitos de Usabilidade}

\subsection{RNF06 - Interface Responsiva}

\textbf{Descrição}: O sistema deve funcionar adequadamente em diferentes dispositivos e tamanhos de tela.

\textbf{Implementação}:
\begin{itemize}
    \item Interface adaptável para desktop e mobile
    \item Elementos visíveis e clicáveis em telas pequenas
    \item Layout organizado e funcional
\end{itemize}

\textbf{Critério de Aceitação}: Interface utilizável tanto em computadores quanto em dispositivos móveis. Todos os elementos são acessíveis.

\subsection{RNF07 - Mensagens Claras}

\textbf{Descrição}: O sistema deve fornecer feedback claro sobre o resultado das operações.

\textbf{Implementação}:
\begin{itemize}
    \item Mensagens de erro compreensíveis em português
    \item Confirmação de operações bem-sucedidas
    \item Indicação visual de carregamento quando necessário
\end{itemize}

\textbf{Critério de Aceitação}: Usuário recebe feedback claro sobre sucesso ou falha de suas ações. Mensagens são compreensíveis.

\section{Requisitos de Confiabilidade}

\subsection{RNF08 - Tratamento de Erros}

\textbf{Descrição}: O sistema deve tratar erros de forma adequada sem quebrar ou perder dados.

\textbf{Implementação}:
\begin{itemize}
    \item Validação de dados antes de salvar no banco
    \item Tratamento de erros de conexão e operações
    \item Manutenção da integridade dos dados
\end{itemize}

\textbf{Critério de Aceitação}: Sistema não quebra com entradas inválidas. Dados permanecem consistentes mesmo após erros.

\section{Requisitos de Portabilidade}

\subsection{RNF09 - Compatibilidade de Sistema}

\textbf{Descrição}: O sistema deve funcionar nos principais sistemas operacionais.

\textbf{Implementação}:
\begin{itemize}
    \item Backend compatível com Windows, Linux e macOS
    \item Frontend compatível com navegadores modernos
    \item Dependências multiplataforma
\end{itemize}

\textbf{Critério de Aceitação}: Sistema executa corretamente em Windows, Linux e macOS. Interface funciona nos principais navegadores.

\section{Requisitos de Manutenibilidade}

\subsection{RNF10 - Organização do Código}

\textbf{Descrição}: O código deve ser organizado de forma clara para facilitar manutenção.

\textbf{Implementação}:
\begin{itemize}
    \item Separação clara entre frontend e backend
    \item Organização em módulos por funcionalidade
    \item Código limpo e bem estruturado
\end{itemize}

\textbf{Critério de Aceitação}: Código é facilmente compreensível e modificável. Estrutura de arquivos é lógica e organizada.


\end{document}